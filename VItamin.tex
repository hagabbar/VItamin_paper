%% Template for a preprint Letter or Article for submission
%% to the journal Nature.
%% Written by Peter Czoschke, 26 February 2004
%%

\documentclass{nature}

%% make sure you have the nature.cls and naturemag.bst files where
%% LaTeX can find them

\bibliographystyle{naturemag}

\title{Estimating Bayesian Parameter Estimation Using Conditional Variational Autoencoders}

%% Notice placement of commas and superscripts and use of &
%% in the author list

\author{Hunter Gabbard$^{1}$, Ik Siong Heng$^1$, Chris Messenger$^1$, Francesco Tonolini$^2$, \& Roderick Murray-Smith$^2$}


\begin{document}

\maketitle

\begin{affiliations}
 \item SUPA, School of Physics and Astronomy, University of Glasgow, Glasgow G12 8QQ, United Kingdom
 \item School of Computing Science, University of Glasgow, Glasgow G12 8QQ, United Kingdom
\end{affiliations}

\begin{abstract}
With the beginning of the Laser Interferometer Gravitational wave Observatory's (LIGO) and Virgo's third observation run well under way, we are now in an era where gravitational wave (GW) detection is commonplace. As the sensitivity of both detectors increases, we will see more many more detections on a weekly and even daily basis.  The current method used to estimate the parameters of gravitational wave events is done using a form of Bayesian inference. Although effective, Bayesian inference is a computationally expensive method which can take of order hours to weeks to complete when applied to a single GW event. We propose the use of a conditional variational autoencoder (CVAE) as a computationally inexpensive alternative to this approach. Here we show that a CVAE can return the posterior estimate for any parameter of a detected GW event on the order of microseconds, an immense speed-up over current inference techniques.

%For Nature, the abstract is really an introductory paragraph set
%in bold type.  This paragraph must be ``fully referenced'' and
%less than 180 words for Letters.  This is the thing that is
%supposed to be aimed at people from other disciplines and is
%arguably the most important part to getting your paper past the
%editors.  End this paragraph with a sentence like ``Here we
%show...'' or something similar.
\end{abstract}

When performing Bayesian GW parameter estimation, we assume that we are given some observed waveform which is buried in noise. For this study, we consider the noise to be whitened Gaussian noise. Given a noisy waveform, we would like to find an optimal procedure for retrieving some finite set of unknown GW parameters \cite{Jaranowski2012}. Our procedure should be able to give us an accurate estimate of the parameters of our observed signal, while also accounting for the uncertainty which comes from having multiple noise realizations of our observed data able to be mapped to one parameter estimate.

According to Bayes Theorem, a posterior for a set of GW parameters can be described by the following expression:

\begin{equation}
    p(\theta|x) = \frac{p(x|\theta) * p(\theta)}{p(x)},
\end{equation}

where $p(\theta|x)$ is the probability of the parameters given observed data, $p(x|\theta)$ is the probability of the observed data given the parameters, $p(\theta)$ is the prior we put on our parameter distribution and $p(x)$ is the probability of our data. We typically assume that $p(x)$ is a constant, $1$, which then reduces to the following equation:

\begin{equation}
    p(\theta|x) \propto p(x|\theta) * p(\theta),
\end{equation}

where $p(\theta|x)$ is the posterior, $p(x|\theta)$ is the likelihood and $p(\theta)$ is the prior. For this study, we assume that the noise is stationary with zero mean and some constant variance in each detector. Small changes in the power spectrum density over time are not considered in this analysis. 

There are several algorithms which may be used to sample from the posterior distribution of astrophysical GW source parameters. The algorithm which is used in our studies is the nested sampling algorithm. Nested sampling takes a multi-dimensional evidence integral calculation (fully marginalized likelihood) and transforms it into a more manageable 1-D integral. Where the fully marginalized likelihood is equivalent to taking the integral of the likelihood and multiplying it by the prior \cite{1409.7215}.

% should state exactly HOW the likelihood is calculated below

The first step of the nested sampling algorithm starts by generating an initial set of live points made from the prior distribution. The likelihood for each point is calculated and the point with the lowest likelihood is removed. The removed sample is then replaced with a new sample which has a higher likelihood. This cycle repeats itself until a predefined stopping threshold is achieved \cite{1409.7215}. Samples from the posterior may be drawn by randomly selecting from both all current 'live' points and all previously removed 'live' points. 

The codebase which we use to do our Bayesian analysis is the Bilby inference library \cite{1811.02042}. For testing, waveforms have a duration of 1 second, sampling frequency of 128Hz, fixed right ascension ($\alpha$), declination ($\delta$), inclination angle ($\theta_j$), spin, luminosity distance, and polarization angle ($\psi$). We allow 3 parameters to vary: chirp mass, phase and time of coalescence. Our waveform model is \texttt{IMRPhenomPv2} with a minimum cutoff frequency of 20Hz. Analysis is done using a single detector (H1) which has a power spectrum density derived from the Advanced LIGO design sensitivity curves.

The priors we choose for the analysis are all fixed for the parameters which we do not allow to vary in our test set. For other parameters, we set a prior on chirp mass from 30 - 43 solar masses, a phase prior of $0 - 2\pi$, distance prior of 1Gpc - 3Gpc, and a time of coalesence prior of $1126259642.4 - 1126259642.6$. The nested sampler is run using 1000 live points and has a predefined stopping criteria. The sampler takes $\mathcal{O}(3 \: \textrm{minutes})$ to converge.

After the sampler has converged, we draw samples and produce a posterior for our astrophysical GW source parameters we are trying to estimate, an example of which can be seen for one test event in Fig. \ref{fig:bilby_pos_ex}. We will now investigate whether we can reproduce the results seen in Fig. \ref{fig:bilby_pos_ex} using our proposed machine learning approach (CVAEs). The advantages and reasons for using such an alternative approach will become more clear in the following sections. 

\begin{figure}
    \centering
    \includegraphics[width=\textwidth]{figures/samp_8_corner.png}
    \caption{Predicted posterior distributions produced by the Bilby inference library for a test GW event with chirp mass $31.21$, phase $2.04$ and time of coalescence $1126259642.5$. Three contours are plotted in the 2D histograms at 1,2 and 3 sigma credibility intervals. 1D histograms have 1 sigma vertical confidence intervals plotted as blue dashed lines.}
    \label{fig:bilby_pos_ex}
\end{figure}
   

%
% intro to conditional variational autoencoders
%

\textit{Conditional Variational Autoencoders}---

To-do:

\begin{itemize}
\item Add Section describing how conditional variational autoencoders work.
\end{itemize}


Then the body of the main text appears after the intro paragraph.
Figure environments can be left in place in the document.
\verb|\includegraphics| commands are ignored since Nature wants
the figures sent as separate files and the captions are
automatically moved to the end of the document (they are printed
out with the \verb|\end{document}| command. However, tables must
be manually moved to the end of the document, after the addendum.

Citation of Einstein's paper \cite{Einstein}.

\begin{figure}
%%%\includegraphics{something} % this command will be ignored
\caption{Each figure legend should begin with a brief title for
the whole figure and continue with a short description of each
panel and the symbols used. For contributions with methods
sections, legends should not contain any details of methods, or
exceed 100 words (fewer than 500 words in total for the whole
paper). In contributions without methods sections, legends should
be fewer than 300 words (800 words or fewer in total for the whole
paper).}
\end{figure}

\section*{Another Section}

Sections can only be used in Articles.  Contributions should be
organized in the sequence: title, text, methods, references,
Supplementary Information line (if any), acknowledgements,
interest declaration, corresponding author line, tables, figure
legends.

Spelling must be British English (Oxford English Dictionary)

In addition, a cover letter needs to be written with the
following:
\begin{enumerate}
 \item A 100 word or less summary indicating on scientific grounds
why the paper should be considered for a wide-ranging journal like
\textsl{Nature} instead of a more narrowly focussed journal.
 \item A 100 word or less summary aimed at a non-scientific audience,
written at the level of a national newspaper.  It may be used for
\textsl{Nature}'s press release or other general publicity.
 \item The cover letter should state clearly what is included as the
submission, including number of figures, supporting manuscripts
and any Supplementary Information (specifying number of items and
format).
 \item The cover letter should also state the number of
words of text in the paper; the number of figures and parts of
figures (for example, 4 figures, comprising 16 separate panels in
total); a rough estimate of the desired final size of figures in
terms of number of pages; and a full current postal address,
telephone and fax numbers, and current e-mail address.
\end{enumerate}

See \textsl{Nature}'s website
(\texttt{http://www.nature.com/nature/submit/gta/index.html}) for
complete submission guidelines.

\begin{methods}
Put methods in here.  If you are going to subsection it, use
\verb|\subsection| commands.  Methods section should be less than
800 words and if it is less than 200 words, it can be incorporated
into the main text.

\subsection{Method subsection.}

Here is a description of a specific method used.  Note that the
subsection heading ends with a full stop (period) and that the
command is \verb|\subsection{}| not \verb|\subsection*{}|.

\end{methods}

%% Put the bibliography here, most people will use BiBTeX in
%% which case the environment below should be replaced with
%% the \bibliography{} command.

% \begin{thebibliography}{1}
% \bibitem{dummy} Articles are restricted to 50 references, Letters
% to 30.
% \bibitem{dummyb} No compound references -- only one source per
% reference.
% \end{thebibliography}

\bibliographystyle{naturemag}
\bibliography{sample}


%% Here is the endmatter stuff: Supplementary Info, etc.
%% Use \item's to separate, default label is "Acknowledgements"

\begin{addendum}
 \item Put acknowledgements here.
 \item[Competing Interests] The authors declare that they have no
competing financial interests.
 \item[Correspondence] Correspondence and requests for materials
should be addressed to Hunter Gabbard~(email: h.gabbard.1@research.gla.ac.uk).
\end{addendum}

%%
%% TABLES
%%
%% If there are any tables, put them here.
%%

\begin{table}
\centering
\caption{This is a table with scientific results.}
\medskip
\begin{tabular}{ccccc}
\hline
1 & 2 & 3 & 4 & 5\\
\hline
aaa & bbb & ccc & ddd & eee\\
aaaa & bbbb & cccc & dddd & eeee\\
aaaaa & bbbbb & ccccc & ddddd & eeeee\\
aaaaaa & bbbbbb & cccccc & dddddd & eeeeee\\
1.000 & 2.000 & 3.000 & 4.000 & 5.000\\
\hline
\end{tabular}
\end{table}

\end{document}
